\documentclass[
14pt,
candidate, % document type
subf, % use and configure subfig package for nested figure numbering
]{article}
% ------------------- Language and Encoding -------------------
\usepackage[T2A]{fontenc}             % Font encoding for Cyrillic (T2A is needed for Ukrainian)
\usepackage[utf8]{inputenc}          % UTF-8 input encoding
\usepackage[english, ukrainian]{babel}        % Language-specific rules (e.g., hyphenation, captions)

% ------------------- Page Layout and Graphics -------------------
\usepackage[left=3cm, right=1.5cm, top=2cm, bottom=2cm]{geometry} % Set page margins
\usepackage{graphicx}                % Include images (JPG, PNG, PDF)
\usepackage{pdfpages}                % Insert external PDF pages
\usepackage[fontsize=14pt]{fontsize}

% ------------------- Tables -------------------
\usepackage{array} % расширенные возможности для работы с таблицами
\usepackage{tabularx} % автоматический подбор ширины столбцов
\usepackage{dcolumn} % выравнивание чисел по разделителю

% ------------------- Mathematics -------------------
\usepackage{bm}                      % Bold math symbols
\usepackage{amsmath}                % Advanced math formatting
\usepackage{amssymb}                % Additional math symbols

% ------------------- Citations and Hyperlinks -------------------
\usepackage{cite}                   % Improved citation handling
\usepackage{hyperref}               % Make references/citations clickable
\hypersetup{
  colorlinks=true,
  linkcolor=black, % Make TOC links black
  urlcolor=black,
  citecolor=black
}

% ------------------- Miscellaneous Features -------------------
\usepackage{color}                  % Define and use custom colors
\usepackage{multirow}               % Merge rows in tables
\usepackage{afterpage}             % Defer content until after the current page
\usepackage[font={normal}]{caption} % Customize figure/table captions
\captionsetup[figure]{format=hang,labelsep=period} % Figure caption formatting
\captionsetup[table]{format=hang,labelsep=period}  % Table caption formatting

\usepackage[onehalfspacing]{setspace} % Set 1.5 line spacing for better readability

\usepackage{xcolor}
\definecolor{wordgray}{gray}{0.5}  % Adjust the 0.5 for lighter/darker gray

\usepackage{fancyhdr}               % Custom headers/footers
\pagestyle{fancy}
\fancyhf{}                          % Clear all header/footer fields
\fancyhead[R]{\textcolor{wordgray}{\normalsize{\thepage}}}             % Page number centered in footer
\renewcommand{\headrulewidth}{0pt}  % Remove top header line

\usepackage{listings}               % Display source code (e.g., C++, Python)

% ------------------- Custom Column Type -------------------
\newcommand{\PreserveBackslash}[1]{\let\temp=\\#1\let\\=\temp} % Ensure \\ works in custom cells
\newcolumntype{C}[1]{>{\PreserveBackslash\centering}p{#1}}     % Define centered column of width #1

% ------------------- Document Structure -------------------
\setcounter{tocdepth}{2}            % Table of contents includes up to subsections
\graphicspath{{images/}}            % Set image search path to "images/" folder

% ------------------- Figures and Diagrams -------------------
\usepackage{float}                  % More precise figure/table placement
\usepackage{pgfplots}               % Create high-quality plots
\pgfplotsset{compat=newest}         % Enable newest features of pgfplots
\usetikzlibrary{positioning,arrows.meta} % TikZ libraries for layout and arrows
\tikzset{
  dot/.style = {                    % Style for small filled circles
    circle, fill,
    minimum size=#1,
    inner sep=0pt, outer sep=0pt
  },
  dot/.default = 5pt                % Default dot size
}
\usepackage{forest}                 % Draw trees and hierarchical diagrams

% ------------------- APA-style bibliography -------------------
% \usepackage{apacite}

% ------------------- table of content stylization -------------------
\usepackage{tocloft}
\usepackage{etoolbox}
\renewcommand{\cftaftertoctitle}{\hfill}
\providecommand{\contentsname}{ЗМІСТ}

\makeatletter
\renewcommand{\tableofcontents}{%
  % \cleardoublepage
  % \thispagestyle{fancy}
  \section*{\tableofcontents}
  \vspace{10em}
  \@starttoc{toc}
}{}{}
\makeatother

% Depth of TOC (to include subsubsection)
\setcounter{tocdepth}{3}

% Indentation for section levels
\cftsetindents{section}{0em}{2em}
\cftsetindents{subsection}{2em}{3em}
\cftsetindents{subsubsection}{4em}{4em}

% Page number alignment and dots
\renewcommand{\cftdotsep}{1} % Controls dot spacing
\renewcommand{\cftsecdotsep}{1}
\renewcommand{\cftsecfont}{\MakeUppercase}
\renewcommand{\cftsubsecfont}{\normalfont}
\renewcommand{\cftsubsubsecfont}{\normalfont}
\renewcommand{\cftsecpagefont}{\normalfont}
\renewcommand{\cftsubsecpagefont}{\normalfont}
\renewcommand{\cftsubsubsecpagefont}{\normalfont}

% Optional: page number alignment (if needed more space)
\renewcommand{\cftsecnumwidth}{5.2em}
% \renewcommand{\cftsubsecnumwidth}{4em}
% \renewcommand{\cftsubsubsecnumwidth}{5em}

% ------------------- Section formatting -------------------
\usepackage{titlesec}
\usepackage{indentfirst}

\setlength{\parindent}{1.25cm}
\setlength{\parskip}{0pt}

% --- Оформлення розділів (18 pt, жирним, прописними) ---
\titleformat{\section}
  [block]
  {\centering\bfseries\fontsize{18pt}{20pt}\selectfont}
  {\thesection}
  {1em}
  {\MakeUppercase}

\titlespacing*{\section}{0pt}{2em plus 1ex minus 0.2ex}{2em plus 1ex}

% --- Оформлення підрозділів (16 pt, жирним) ---
\titleformat{\subsection}
  [block]
  {\bfseries\fontsize{16pt}{18pt}\selectfont}
  {\thesubsection}
  {1em}
  {}

\titlespacing*{\subsection}{\parindent}{2em plus 1ex minus 0.2ex}{2em plus 1ex}

% --- Оформлення пунктів (14 pt, курсив) ---
\setcounter{secnumdepth}{3}

\titleformat{\subsubsection}
  [block]
  {\itshape\fontsize{14pt}{16pt}\selectfont}
  {\thesubsubsection}
  {1em}
  {}

\titlespacing*{\subsubsection}{\parindent}{2em plus 1ex minus 0.2ex}{2em plus 0.5ex}

% --- Нумерація ---
\renewcommand{\thesection}{РОЗДІЛ \arabic{section}}
\renewcommand{\thesubsection}{\arabic{section}.\arabic{subsection}}
\renewcommand{\thesubsubsection}{\thesubsection.\arabic{subsubsection}}


\def\thebibliography#1{\section*{Перелік посилань}\eightpt\list
 {\arabic{enumi}.}{\settowidth\labelwidth{[#1]}\leftmargin\labelwidth
 \advance\leftmargin\labelsep
 \usecounter{enumi}}
 \def\newblock{\hskip .11em plus .33em minus .07em}
 \sloppy\clubpenalty4000\widowpenalty4000
 \sfcode`\.=1000\relax}
\let\endthebibliography=\endlist

\begin{document}


% Титульний аркуш
\begin{titlepage}
\thispagestyle{empty}
\begin{spacing}{1}

\begin{center}
\textbf{НАЦІОНАЛЬНИЙ ТЕХНІЧНИЙ УНІВЕРСИТЕТ УКРАЇНИ}\\
\textbf{«КИЇВСЬКИЙ ПОЛІТЕХНІЧНИЙ ІНСТИТУТ\\
імені ІГОРЯ СІКОРСЬКОГО»}\\

\textbf{Навчально-науковий інститут прикладного системного аналізу\\
Кафедра штучного інтелекту}
\end{center}

\vspace{0.1cm}

\hfill
\begin{minipage}{0.4\textwidth}
До захисту допущено:\\
В.о. завідувачки кафедри\\
\underline{\hspace{2cm}} Ірина ДЖИГИРЕЙ\\
«\underline{\hspace{0.7cm}}» \underline{\hspace{2.8cm}} 2025 р.
\end{minipage}

\vspace{0.1cm}

\begin{center}
\textbf{\LARGE Дипломна робота}\\ 

\textbf{на здобуття ступеня бакалавра\\
за освітньо-професійною програмою «Системи і методи штучного інтелекту»\\
спеціальності 122 «Комп’ютерні науки»\\
на тему: «\textcolor{red}{Тема}»}
\end{center}

\vspace{0.05cm}

\noindent
Виконав:\\
студент IV курсу, групи \textcolor{red}{Номер групи}\\
\textcolor{red}{ПІБ} \hfill \underline{\hspace{4cm}}

\vspace{1ex}

\noindent
Керівник:\\
\textcolor{red}{Посада, науковий ступінь, вчене звання}\\
\textcolor{red}{ПІБ} \hfill \underline{\hspace{4cm}}

\vspace{1ex}

\noindent
Консультант з економічного розділу:\\
доцент кафедри економічної кібернетики, к.е.н., доцент,\\
Рощина Надія Василівна \hfill \underline{\hspace{4cm}}

\vspace{1ex}

\noindent
Консультант з нормоконтролю:\\
фахівець першої категорії кафедри штучного інтелекту, к.т.н., доцент\\
Комариста Богдана Миколаївна \hfill \underline{\hspace{4cm}}

\vspace{1ex}

\noindent
Рецензент:\\
\textcolor{red}{Посада, науковий ступінь, вчене звання}\\
\textcolor{red}{Прізвище Ім’я По батькові} \hfill \underline{\hspace{4cm}}


\vspace{0.5cm}


\hfill
\begin{minipage}{0.5\textwidth}
Засвідчую, що у цій дипломній роботі немає запозичень з праць інших авторів без відповідних посилань.\\
Студент \underline{\hspace{5cm}}
\end{minipage}

\begin{center}
Київ -- 2025 року
\end{center}
\end{spacing}
\end{titlepage}


% \documentclass[../diploma]{subfiles}

% \begin{document}

% \begin{titlepage}
% \vspace*{-4em}\linespread{1.0}\selectfont
% \centering\bf

% НАЦІОНАЛЬНИЙ ТЕХНІЧНИЙ УНІВЕРСИТЕТ УКРАЇНИ <<КИЇВСЬКИЙ ПОЛІТЕХНІЧНИЙ ІНСТИТУТ\\
% імені ІГОРЯ СІКОРСЬКОГО>>\\
% НАВЧАЛЬНО-НАУКОВИЙ ІНСТИТУТ ПРИКЛАДНОГО СИСТЕМНОГО АНАЛІЗУ

% \medbreak

% Кафедра математичних методів системного аналізу

% \bigbreak

% \begin{tblr}{
% 	columns={colsep=0pt,font=\normalfont},
% 	colspec={X l},
% }
% & До захисту допущено:\\
% & Завідувач кафедри\\
% & \fillin[1.5cm] Оксана ТИМОЩУК\\
% & "\fillin{16}"\ \fillin{травня} 2024 р. % use \fillin*{травня} to actually show text
% \end{tblr}

% \vfill

% {\Large Дипломна робота}\\[1ex]
% на здобуття ступеня бакалавра\\
% \makebox[0pt]{за освітньо-професійною програмою <<Системний аналіз і управління>>}\\
% спеціальності 124 <<Системний аналіз>>\\
% на тему: <<Здесь может быть ваша реклама>>

% \vfill

% \normalfont

% \linespread{0.4}\selectfont
% \begin{tblr}{
% 	columns={colsep=0pt},
% 	colspec={X r},
% 	cell{Z}{1}={c=2}{l},
% 	cell{Z}{2}={l,wd=7.5cm}
% }
% Виконав: & \\
% Студент IV курсу, групи КА-XX & \\
% Іванов Іван Іванович & \fillin[3cm] \\
% \\
% Керівник: & \\
% доц., к.ф.-м.н. Петренко Петро Петрович & \fillin[3cm] \\
% \\
% Консультант з економічного розділу: & \\
% к.е.н. Сидоренко Сидір Сидорович & \fillin[3cm] \\
% \\
% Консультант з нормоконтролю: & \\
% к.ф.-м.н., Миколенко Микола Миколайович & \fillin[3cm] \\
% \\
% Рецензент: & \\
% доц., к. ф.-м.н. Іваненко Івана Іванівна & \fillin[3cm] \\
% \\
% \end{tblr}
% \begin{tblr}{X Q[l,wd=7.5cm]}
% &	{\linespread{0.9}\selectfont
% 		Засвідчую, що у цій дипломній роботі немає запозичень з праць інших авторів без відповідних посилань.\\
% 		Студент \fillin[3cm]
% 	}
% \end{tblr}

% \vfill

% Київ --- 2024 року
% \end{titlepage}
% \end{document}

% Завдання
\thispagestyle{empty}
\begin{center}
\textbf{
Національний технічний університет України\\
«Київський політехнічний інститут імені Ігоря Сікорського»\\
Навчально-науковий інститут прикладного системного аналізу\\
Кафедра штучного інтелекту\\
}
\end{center}

\noindent
Рівень вищої освіти – перший (бакалаврський)\\
Спеціальність -- 122 «Комп'ютерні науки»\\
Освітньо-професійна програма «Системи і методи штучного інтелекту»\\

\vspace{1em}

\hfill
\begin{minipage}{0.4\textwidth}
ЗАТВЕРДЖУЮ\\
В.о. завідувачки кафедри\\
\underline{\hspace{2cm}} Ірина ДЖИГИРЕЙ\\
«15» січня 2025 р.
\end{minipage}

\vspace{2em}

\begin{center}
    \textbf{ЗАВДАННЯ\\на дипломну роботу студенту\\\textcolor{red}{ПІБ}}
\end{center}

\newcounter{boxlblcounter}  
\newcommand{\makeshiftedemun}[1]{#1.}% \hfill fills the label box
\newenvironment{shiftedemun}
  {\begin{list}
    {\arabic{boxlblcounter}}
    {\usecounter{boxlblcounter}
     \setlength{\labelwidth}{-0.5em}
     \setlength{\labelsep}{0.5em}
     \setlength{\itemsep}{2pt}
     \setlength{\leftmargin}{0cm}
     \setlength{\rightmargin}{0cm}
     \setlength{\itemindent}{0em} 
     \let\makelabel=\makeshiftedemun
    }
  }
{\end{list}}

\begin{shiftedemun}
    \item 
    Тема роботи «\textcolor{red}{Тема}», керівник роботи \textcolor{red}{ПІБ}, \textcolor{red}{науковий ступінь, вчене звання}, затверджені наказом по університету від «\underline{\hspace{0.7cm}}» \underline{\hspace{2.8cm}} 2025 р. № \underline{\hspace{1.5cm}}
    
    \item
    Термін подання студентом роботи «09» червня 2025 року.
    
    \item
    \textcolor{red}{Вихідні дані до роботи} 

    \item
    \textcolor{red}{Зміст роботи} 

    \item
    \textcolor{red}{Перелік ілюстративного матеріалу (із зазначенням плакатів, презентацій тощо)} 

    \item
    Консультанти розділів роботи\\
    \noindent % Prevents indentation for the tabular environment
    \begin{tabular}{|m{0.2\textwidth}|m{0.4\textwidth}|m{0.15\textwidth}|m{0.15\textwidth}|} % m{} for vertical alignment in cells
        \hline
        \centering\arraybackslash \textbf{Розділ} & \centering\arraybackslash \textbf{Прізвище, ініціали та посада консультанта} & \multicolumn{2}{c|}{\centering\arraybackslash \textbf{Підпис, дата}} \\
        \cline{3-4} % Partial line under the multicolumn
        & & \centering\arraybackslash завдання видав & \centering\arraybackslash завдання прийняв \\
        \hline
        % Example row from the image
        \centering\arraybackslash Економічний & \textcolor{red}{Рощина Надія Василівна, доцент, к. е. н.} & \centering\arraybackslash  & \centering\arraybackslash  \\
        \hline
    \end{tabular}

    \item
    Дата видачі завдання «03» лютого 2025 року.\\
    \begin{table}[h!]
    \caption*{Календарний план}
    \begin{tabular}{|m{0.04\textwidth}|m{0.5\textwidth}|m{0.21\textwidth}|m{0.15\textwidth}|}
        
        \hline
        \centering\arraybackslash \small{№ з/п} & \centering\arraybackslash Назва етапів виконання дипломної роботи & \centering\arraybackslash Термін виконання етапів роботи & \centering\arraybackslash Примітка\\
        \hline
        \centering\arraybackslash 1 & \centering\arraybackslash  & \centering\arraybackslash 21.04.2025 & \centering\arraybackslash Виконано \\
        \hline
        \centering\arraybackslash 2 & \centering\arraybackslash  & \centering\arraybackslash 28.04.2025 & \centering\arraybackslash Виконано \\
        \hline
        \centering\arraybackslash 3 & \centering\arraybackslash  & \centering\arraybackslash 05.05.2025 & \centering\arraybackslash Виконано \\
        \hline
        \centering\arraybackslash 4 & \centering\arraybackslash  & \centering\arraybackslash 12.05.2025 & \centering\arraybackslash Виконано \\
        \hline
        \centering\arraybackslash 5 & \centering\arraybackslash  & \centering\arraybackslash 19.05.2025 & \centering\arraybackslash Виконано \\
        \hline
        \centering\arraybackslash 6 & \centering\arraybackslash  & \centering\arraybackslash 26.05.2025 & \centering\arraybackslash Виконано \\
        \hline
        \centering\arraybackslash 7 & \centering\arraybackslash  & \centering\arraybackslash 02.06.2025 & \centering\arraybackslash Виконано \\
        \hline
        \centering\arraybackslash 8 & \centering\arraybackslash  & \centering\arraybackslash 09.06.2025 & \centering\arraybackslash Виконано \\
        \hline
        \end{tabular}
    \end{table}
\end{shiftedemun}

Студент \hfill \textcolor{red}{Власне ім’я, ПРІЗВИЩЕ}

Керівник \hfill \textcolor{red}{Власне ім’я, ПРІЗВИЩЕ}

\thispagestyle{empty}


% Реферат
\newpage
\begin{center}
    \textbf{РЕФЕРАТ}
\end{center} 

Дипломна робота: ХХ с., ХХ рис., ХХ табл., ХХ посилань, додаток.

КЛЮЧОВЕ СЛОВО 1, КЛЮЧОВЕ СЛОВО 2, КЛЮЧОВЕ СЛОВО 3, КЛЮЧОВЕ СЛОВО 4, КЛЮЧОВЕ СЛОВО 5, КЛЮЧОВЕ СЛОВО 6, КЛЮЧОВЕ СЛОВО 7, КЛЮЧОВЕ СЛОВО 8, КЛЮЧОВЕ СЛОВО 9, КЛЮЧОВЕ СЛОВО 10.

Об’єктом дослідження 

Предметом дослідження 

Метою роботи 

\newpage

\begin{center}
    \textbf{ABSTRACT}
\end{center} 

Master's thesis: ХХ p., ХХ figures, ХХ tables, ХХ references, appendix.

KEYWORD 1, ...

The object of the study is ...

The subject of research is ...

The purpose of the work is to ...

\newpage


% Зміст
\tableofcontents

\newpage

% Перелік умовних позначень
% \section*{Перелік умовних позначень, скорочень і термінів}
% \addcontentsline{toc}{section}{Перелік умовних позначень, скорочень і термінів}
% ...

% Вступ
\section*{\centering ВСТУП}
\addcontentsline{toc}{section}{ВСТУП}



\newpage
...

% Основна частина
\section{Introduction}

Оптимізація кольорової палітри є важливою задачею в сучасній обробці
зображень, яка знаходить застосування у різних галузях, таких як комп'ютерна
графіка, цифрова фотографія, медична візуалізація, геоінформаційні системи, а
також у розробці інтерфейсів користувача для пристроїв з обмеженими
ресурсами. У зв'язку з швидким розвитком технологій та зростанням кількості
цифрових зображень, потреба у ефективних методах оптимізації кольорової
палітри стає все більш актуальною.

\subsection{Motivation}

\cite{linal}

\subsubsection{Details}

\subsubsection{Details}

\subsection{Analysis}

...

\newpage

% Висновки
\section*{Висновки}
\addcontentsline{toc}{section}{Висновки}

\newpage

% Список використаних джерел
\bibliographystyle{apacite}
\bibliography{mybib}

\newpage

% Додатки
\appendix
\section*{Додаток А}
\addcontentsline{toc}{section}{Додаток А}
...

\end{document}
