% ------------------- Language and Encoding -------------------
\usepackage[T2A]{fontenc}             % Font encoding for Cyrillic (T2A is needed for Ukrainian)
\usepackage[utf8]{inputenc}          % UTF-8 input encoding
\usepackage[english, ukrainian]{babel}        % Language-specific rules (e.g., hyphenation, captions)

% ------------------- Page Layout and Graphics -------------------
\usepackage[left=3cm, right=1.5cm, top=2cm, bottom=2cm]{geometry} % Set page margins
\usepackage{graphicx}                % Include images (JPG, PNG, PDF)
\usepackage{pdfpages}                % Insert external PDF pages
\usepackage[fontsize=14pt]{fontsize}

% ------------------- Tables -------------------
\usepackage{array} % расширенные возможности для работы с таблицами
\usepackage{tabularx} % автоматический подбор ширины столбцов
\usepackage{dcolumn} % выравнивание чисел по разделителю

% ------------------- Mathematics -------------------
\usepackage{bm}                      % Bold math symbols
\usepackage{amsmath}                % Advanced math formatting
\usepackage{amssymb}                % Additional math symbols

% ------------------- Citations and Hyperlinks -------------------
\usepackage{cite}                   % Improved citation handling
\usepackage{hyperref}               % Make references/citations clickable
\hypersetup{
  colorlinks=true,
  linkcolor=black, % Make TOC links black
  urlcolor=black,
  citecolor=black
}

% ------------------- Miscellaneous Features -------------------
\usepackage{color}                  % Define and use custom colors
\usepackage{multirow}               % Merge rows in tables
\usepackage{afterpage}             % Defer content until after the current page
\usepackage[font={normal}]{caption} % Customize figure/table captions
\captionsetup[figure]{format=hang,labelsep=period} % Figure caption formatting
\captionsetup[table]{format=hang,labelsep=period}  % Table caption formatting

\usepackage[onehalfspacing]{setspace} % Set 1.5 line spacing for better readability

\usepackage{xcolor}
\definecolor{wordgray}{gray}{0.5}  % Adjust the 0.5 for lighter/darker gray

\usepackage{fancyhdr}               % Custom headers/footers
\pagestyle{fancy}
\fancyhf{}                          % Clear all header/footer fields
\fancyhead[R]{\textcolor{wordgray}{\normalsize{\thepage}}}             % Page number centered in footer
\renewcommand{\headrulewidth}{0pt}  % Remove top header line

\usepackage{listings}               % Display source code (e.g., C++, Python)

% ------------------- Custom Column Type -------------------
\newcommand{\PreserveBackslash}[1]{\let\temp=\\#1\let\\=\temp} % Ensure \\ works in custom cells
\newcolumntype{C}[1]{>{\PreserveBackslash\centering}p{#1}}     % Define centered column of width #1

% ------------------- Document Structure -------------------
\setcounter{tocdepth}{2}            % Table of contents includes up to subsections
\graphicspath{{images/}}            % Set image search path to "images/" folder

% ------------------- Figures and Diagrams -------------------
\usepackage{float}                  % More precise figure/table placement
\usepackage{pgfplots}               % Create high-quality plots
\pgfplotsset{compat=newest}         % Enable newest features of pgfplots
\usetikzlibrary{positioning,arrows.meta} % TikZ libraries for layout and arrows
\tikzset{
  dot/.style = {                    % Style for small filled circles
    circle, fill,
    minimum size=#1,
    inner sep=0pt, outer sep=0pt
  },
  dot/.default = 5pt                % Default dot size
}
\usepackage{forest}                 % Draw trees and hierarchical diagrams

% ------------------- APA-style bibliography -------------------
\usepackage{apacite}

% ------------------- table of content stylization -------------------
\usepackage{tocloft}
\usepackage{etoolbox}
\renewcommand{\cftaftertoctitle}{\hfill}
\providecommand{\contentsname}{ЗМІСТ}

\makeatletter
\renewcommand{\tableofcontents}{%
  % \cleardoublepage
  % \thispagestyle{fancy}
  \section*{\MakeUppercase \tableofcontents}
  \vspace{10em}
  \@starttoc{toc}
}{}{}
\makeatother

% Depth of TOC (to include subsubsection)
\setcounter{tocdepth}{3}

% Indentation for section levels
\cftsetindents{section}{0em}{2em}
\cftsetindents{subsection}{2em}{3em}
\cftsetindents{subsubsection}{4em}{4em}

% Page number alignment and dots
\renewcommand{\cftdotsep}{1} % Controls dot spacing
\renewcommand{\cftsecdotsep}{1}
\renewcommand{\cftsecfont}{\normalfont}
\renewcommand{\cftsubsecfont}{\normalfont}
\renewcommand{\cftsubsubsecfont}{\normalfont}
\renewcommand{\cftsecpagefont}{\normalfont}
\renewcommand{\cftsubsecpagefont}{\normalfont}
\renewcommand{\cftsubsubsecpagefont}{\normalfont}

% Optional: page number alignment (if needed more space)
\renewcommand{\cftsecnumwidth}{5.2em}
% \renewcommand{\cftsubsecnumwidth}{4em}
% \renewcommand{\cftsubsubsecnumwidth}{5em}

% ------------------- Section formatting -------------------
\usepackage{titlesec}
\usepackage{indentfirst}

\setlength{\parindent}{1.25cm}
\setlength{\parskip}{0pt}

% --- Оформлення розділів (18 pt, жирним, прописними) ---
\titleformat{\section}
  [block]
  {\centering\bfseries\fontsize{18pt}{20pt}\selectfont\MakeUppercase}
  {\thesection}
  {1em}
  {}

\titlespacing*{\section}{0pt}{2em plus 1ex minus 0.2ex}{2em plus 1ex}

% --- Оформлення підрозділів (16 pt, жирним) ---
\titleformat{\subsection}
  [block]
  {\bfseries\fontsize{16pt}{18pt}\selectfont}
  {\thesubsection}
  {1em}
  {}

\titlespacing*{\subsection}{\parindent}{2em plus 1ex minus 0.2ex}{2em plus 1ex}

% --- Оформлення пунктів (14 pt, курсив) ---
\setcounter{secnumdepth}{3}

\titleformat{\subsubsection}
  [block]
  {\itshape\fontsize{14pt}{16pt}\selectfont}
  {\thesubsubsection}
  {1em}
  {}

\titlespacing*{\subsubsection}{\parindent}{2em plus 1ex minus 0.2ex}{2em plus 0.5ex}

% --- Нумерація ---
\renewcommand{\thesection}{РОЗДІЛ \arabic{section}}
\renewcommand{\thesubsection}{\arabic{section}.\arabic{subsection}}
\renewcommand{\thesubsubsection}{\thesubsection.\arabic{subsubsection}}
