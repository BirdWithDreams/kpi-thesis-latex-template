\documentclass[../diploma]{subfiles}

\newcommand\mydesclabel[1]{#1\quad{\normalfont ---}}
\setlist[description]{format=\mydesclabel}
\setlist[itemize]{label=---}

\begin{document}

\chapter{Функціонально-вартісний аналіз програмного продукту}

В заданому розділі буде проведено оцінювання основних характеристик для майбутнього програмного продукту, що спеціалізується на дослідженні демографічного стану.

Дана реалізація буде сприяти проведенню усіх необхідних досліджень, що дасть змогу якісно дослідити питання не лише в Україні, проте у всьому світі. 
Також в даному дослідженні показано різні варіанти реалізації для забезпечення найбільш коректної та оптимальної стратегії вибору, що має вплив на економічні фактори та сумісність з майбутнім програмним продуктом. Для цього застосовувався апарат функціонально-вартісного аналізу.

Функціонально-вартісний аналіз (ФВА) передбачає собою технологію, що дозволяє оцінити реальну вартість продукту або послуги незалежно від організаційної структури компанії. ФВА проводиться з метою виявлення резервів зниження витрат за рахунок ефективніших варіантів виробництва, кращого співвідношення між споживчою вартістю виробу та витратами на його виготовлення. Для проведення аналізу використовується економічна, технічна та конструкторська інформація.

Алгоритм функціонально-вартісного аналізу включає в себе визначення послідовності етапів розробки продукту, визначення повних витрат (річних) та кількості робочих часів, визначення джерел витрат та кінцевий розрахунок вартості програмного продукту.

\section{Постановка задачі проектування}

У роботі застосовується метод ФВА для  проведення техніко-економічного аналізу розробки системи прогнозу стійкості фінансових показників. Оскільки рішення стосовно проектування та реалізації компонентів, що розробляється, впливають на всю систему, кожна окрема підсистема має її задовольняти. Тому фактичний аналіз представляє собою аналіз функцій програмного продукту, призначеного для збору, обробки та проведення аналізу даних по компанії.

Технічні вимоги до програмного продукту є наступні:
\begin{itemize}
\item функціонування на персональних комп'ютерах із стандартним набором компонентів;
\item зручність та зрозумілість для користувача;
\item швидкість обробки даних та доступ до інформації в реальному часі;
\item можливість зручного масштабування та обслуговування;
\item мінімальні витрати на впровадження програмного продукту.
\end{itemize}

\section{Обґрунтування функцій програмного продукту}

Головна функція $F_0$ -- розробка можливого програмного продукту, яка дозволяє аналізувати різні характеристики, що безпосередньо впливають на стійкість підприємства.  Беручи за основу цю функцію, можна виділити наступні:
\begin{description}
\item[$F_1$] вибір самої програми.
\item[$F_2$] якісний аналіз даних.
\item[$F_3$] графічні показники.
\end{description}

Кожна з цих функцій має декілька варіантів реалізації:

\begin{itemize}[label={}]
\item Функція $F_1$:
	\begin{enumerate}[label=\alph*)]
	\item Eviews.
	\item R.
	\end{enumerate}
\item Функція $F_2$:
	\begin{enumerate}[label=\alph*)]
	\item Застосування вбудованих функцій.
	\item Створення своїх обчислень значень.
	\end{enumerate}
\item Функція $F_3$:
	\begin{enumerate}[label=\alph*)]
	\item Використання шаблонних графіків.
	\item Створення своїх.
	\end{enumerate}
\end{itemize}

Варіанти реалізації основних функцій  наведені у  морфологічній карті системи (рис. \ref{fig:morph_map}).

\begin{figure}[H]
\centering
\end{figure}
\begin{figure}[H]
\centering
\begin{forest}
for tree={
	draw, edge={semithick,-Stealth},
	s sep+=3em,
	l sep+=1em,
	minimum width=2cm,
	minimum height=1cm,
	align=center
},
e/.style={tikz/.process={Ow{edge}{\draw [##1] () edge (#1);}}},
[,phantom
 [Eviews, e=!r2>
  [Застосування\\вбудованих функцій, e=!r2>>
   [Використання\\шаблонних графіків]
  ]
 ] 
 [R, e=!r1>
  [Створення своїх\\обчислень значень, e=!r1>>
   [Використання\\шаблонних графіків]
  ]
 ]
]
\end{forest}
\vspace{1ex}\caption{Морфологічна карта}\label{fig:morph_map}
\end{figure}

Морфологічна карта відображає множину всіх можливих варіантиів основних функцій. Позитивно-негативна матриця показана в таблиці \ref{tbl:positive-negative}

\begin{longtblr}[
	label=tbl:positive-negative,
	caption={Позитивно-негативна матриця}
]{
	hlines, vlines,
	columns={m,2},
	column{1-2}={c,-1},
	row{1}={c,font=\bfseries},
	cell{2,4,6}{1}={r=2}{},
	rowhead=1
}
Функції & Варіанти реалізації & Переваги & Недоліки\\
$F_1$ & А & Загальнодоступна програма, доступність багатьох бібліотек & Необхідність повної реалізації алгоритму\\*
& Б & Доступна в реалізації програма для різних обчислень & На написання коду необхідно мати базові навички та вміння\\
$F_2$ & А & Доступність та легкість при написанні & Іноді не відповідає задачі яку треба розв’язати\\*
& Б & Ідеально описують усі необхідні характеристики & Достатньо затратно реалізовувати свої алгоритми для подальшої реалізації\\
$F_3$ & А & Загально прийнята реалізація & Іноді не відповідає очікуваним значенням\\*
& Б & При виконанні власних досліджень краще може передавати висновки & Необхідно достатньо багато часу для написання програми для побудови та знаходження всього необхідного в задачі
\end{longtblr}

На основі аналізу позитивно-негативної матриці робимо висновок, що при розробці програмного продукту деякі варіанти реалізації функцій варто відкинути, тому, що вони не відповідають поставленим перед програмним продуктом задачам. Ці варіанти відзначені у морфологічній карті.

\begin{itemize}
\item Функція $F_1$:\par
Перевагу даємо загальнодоступності. Для спрощення роботи по написанню коду варіант Б має бути відкинутий.

\item Функція $F_2$:\par
Програма допускає обрання обох варіантів. Можливо використати варіанти А чи Б.

\item Функція $F_3$:\par
Реалізація першого варіанту є сприйнятливою для програми. Це варіант А.
\end{itemize}

Таким чином, будемо розглядати такий варіанти реалізації ПП:
\begin{gather*}
F_1А - F_2А - F_3А\\
F_1А - F_2Б - F_3А
\end{gather*}

Для оцінювання якості розглянутих функцій обрана система параметрів, описана нижче.

\section{Обґрунтування системи параметрів програмного продукту}

На основі даних, розглянутих вище, визначаються основні параметри вибору, які будуть використані для розрахунку коефіцієнта технічного рівня.

Для того, щоб охарактеризувати програмний продукт, будемо використовувати наступні параметри:
\begin{itemize}
\item $X_1$ -- швидкодія мови програмування;
\item $X_2$ -- об’єм пам’яті для обчислень та збереження даних;
\item $X_3$ -- час навчання даних;
\item $X_4$ -- потенційний об’єм програмного коду.
\end{itemize}

Гірші, середні і кращі значення параметрів вибираються на основі вимог замовника й умов, що характеризують експлуатацію програмного продукту,  як показано у таблиці \ref{tbl:main-specs}

\def\pgfparse#1{\pgfmathparse{#1}\pgfmathresult}

\def\xi{60, 80, 110}
\def\xii{60, 50, 30}
\def\xiii{80, 70, 60}
\def\xiv{35, 25, 20}

\begin{longtblr}[
	label=tbl:main-specs,
	caption={Основні параметри програмного продукту}
]{
	hlines, vlines,
	columns={m,-1},
	cell{1}{1-3}={r=2}{c,font=\bfseries},
	cell{1}{4}={r=1,c=3}{c,font=\bfseries},
	column{2,X-Z}={c,0.4},
	column{2,3}={0.6}
}
Назва Параметра & Умовні позначення & Одиниці виміру & Значення параметра &  &  \\
 &  &  & гірші & середні & кращі \\
Швидкодія мови програмування & $X_1$ & оп/мс & \pgfparse{{\xi}[0]} & \pgfparse{{\xi}[1]} & \pgfparse{{\xi}[2]} \\
 Об’єм пам’яті  & $X_2$ & Мб & \pgfparse{{\xii}[0]} & \pgfparse{{\xii}[1]} & \pgfparse{{\xii}[2]} \\
Час попередньої обробки даних & $X_3$ & мс & \pgfparse{{\xiii}[0]} & \pgfparse{{\xiii}[1]} & \pgfparse{{\xiii}[2]} \\
Потенційний об’єм програмного коду & $X_4$ & кількість рядків коду & \pgfparse{{\xiv}[0]} & \pgfparse{{\xiv}[1]} & \pgfparse{{\xiv}[2]}
\end{longtblr}

За даними таблиці \ref{tbl:main-specs} будуються графічні характеристики параметрів -- рис. \ref{fig:X1}-\ref{fig:X3}.

\pgfplotsset{width=10cm}

\begin{figure}[H]
\centering
\begin{tikzpicture}
\begin{axis}[
	xmin=0,
	xmax=4,
	ymin=min(\xi)-10,
	ymax=max(\xi)+10
]

\addplot (x,x);

\draw[thick, blue] \foreach \p [count=\i,remember=\p as \pl] in \xi {
    \ifnum\i>1\relax
    (\i-1,\pl) node[dot]{} -- (\i,\p) node[dot]{}
    \fi
};

\end{axis}
\end{tikzpicture}
\caption{$X_1$, швидкодія мови програмування}\label{fig:X1}
\end{figure}

\begin{figure}[H]
\centering
\begin{tikzpicture}
\begin{axis}[
	xmin=0,
	xmax=4,
	ymin=min(\xii)-10,
	ymax=max(\xii)+10
]

\addplot (x,x);

\draw[thick, blue] \foreach \p [count=\i,remember=\p as \pl] in \xii {
    \ifnum\i>1\relax
    (\i-1,\pl) node[dot]{} -- (\i,\p) node[dot]{}
    \fi
};

\end{axis}
\end{tikzpicture}
\caption{$X_2$, об'єм пам'яті }\label{fig:X2}
\end{figure}

\begin{figure}[H]
\centering
\begin{tikzpicture}
\begin{axis}[
	xmin=0,
	xmax=4,
	ymin=min(\xiii)-10,
	ymax=max(\xiii)+10
]

\addplot (x,x);

\draw[thick, blue] \foreach \p [count=\i,remember=\p as \pl] in \xiii {
    \ifnum\i>1\relax
    (\i-1,\pl) node[dot]{} -- (\i,\p) node[dot]{}
    \fi
};

\end{axis}
\end{tikzpicture}
\caption{$X_3$, час попередньої обробки даних}\label{fig:X3}
\end{figure}

\begin{figure}[H]
\centering
\begin{tikzpicture}
\begin{axis}[
	xmin=0,
	xmax=4,
	ymin=min(\xiv)-10,
	ymax=max(\xiv)+10
]

\addplot (x,x);

\draw[thick, blue] \foreach \p [count=\i,remember=\p as \pl] in \xiv {
    \ifnum\i>1\relax
    (\i-1,\pl) node[dot]{} -- (\i,\p) node[dot]{}
    \fi
};

\end{axis}
\end{tikzpicture}
\caption{$X_4$, потенційний об'єм програмного коду}\label{fig:X4}
\end{figure}

\section{Аналіз експертного оцінювання параметрів}

Після детального обговорення й аналізу кожний експерт оцінює ступінь важливості кожного параметру для конкретно поставленої цілі -- розробка програмного продукту, який дає найбільш точні результати при знаходженні параметрів моделей адаптивного прогнозування і обчислення прогнозних значень.

Значимість кожного параметра визначається методом попарного порів¬няння. Оцінку проводить експертна комісія із 7 людей. Визначення коефіцієнтів значимості передбачає:
\begin{itemize}
\item визначення рівня значимості параметра шляхом присвоєння різних рангів;
\item перевірку придатності експертних оцінок для подальшого використання;
\item визначення оцінки попарного пріоритету параметрів;
\item обробку результатів та визначення коефіцієнту значимості.
\end{itemize}

Для перевірки степені достовірності експертних оцінок, визначимо наступні параметри:
\begin{enumerate}
\item сума рангів кожного з параметрів і загальна сума рангів:
$$
R_i = \sum_{j=1}^{N}r_{ij}R_{ij} = N\frac{n(n+1)}{2} = 70
$$
де $N$ -- число експертів, $n$ -- кількість параметрів;

\item середня сума рангів:
$$
T = \frac1n R_{ij} = 17,5
$$

\item відхилення суми рангів кожного параметра від середньої суми рангів:
$$
\Delta_i = R_i-T
$$
Сума відхилень по всім параметрам повинна дорівнювати 0;

\item загальна сума квадратів відхилення:
$$
S = \sum_{i=1}^{N}\Delta_i^2 = 197
$$
Порахуємо коефіцієнт узгодженості:
$$
W= \frac{12S}{N^2(n^3-n)} = \frac{12\cdot197}{7^2(4^3-4)} = 0,754 > W_k = 0,67
$$
\end{enumerate}

Ранжування можна вважати достовірним, тому що знайдений коефіцієнт узгодженості перевищує нормативний, котрий дорівнює 0,67.

\bigbreak

Результати експертного ранжування наведені у таблиці \ref{tbl:ranking}

\begin{longtblr}[
	label=tbl:ranking,
	caption={Результати ранжування параметрів}
]{columns={c,wd=\textwidth,cmd=\clap}}
\begin{tblr}{
	width=\textwidth+.8cm,
	hlines, vlines,
	columns={c,m,-1},
	row{1}={font=\bfseries},
	column{1}={2.7},
	column{2}={l,3.5},
	column{3}={l,2.5},
	column{4-W}={m,0.25},
	column{X}={2,colsep=2pt},
	column{Y}={1.8},
	cell{1}{1}={cmd=\clap},
	cell{Z}{4-W}={cmd=\clap},
	cell{2-Z}{Z}={cmd=\clap},
	cell{1}{4}={c=7}{wd=4.1cm},
	cell{1}{1-3, X-Z}={r=2}{c},
	cell{Z}{1}={c=3}{},
}
Параметр & Назва параметра & Одиниці виміру & Ранг параметра за оцінкою експерта  &  &  &  &  &  &  & {Сума рангів $R_i$} & Відхи\-лення $\Delta i$ & $\Delta i^2$ \\
 &  &  & 1 & 2 & 3 & 4 & 5 & 6 & 7 &  &  &  \\
$X_1$ & Швидкодія мови програмування  & Оп/мс & 1 & 2 & 2 & 1 & 1 & 1 & 2 & 10 & -7,5 & 56,25 \\
$X_2$ & Об'єм пам'яті & Mb & 3 & 4 & 3 & 3 & 4 & 3 & 4 & 24 & 6,5 & 42,25 \\
$X_3$ & Час попередньої обробки даних & мс & 2 & 1 & 1 & 2 & 2 & 2 & 1 & 11 & -6,5 & 42,25 \\
$X_4$ & Потенційний об’єм програмного коду & Кількість рядків коду & 4 & 3 & 4 & 4 & 3 & 4 &  3 & 25 & 7,5 & 56,25 \\
Разом  &  &  & 10 & 10 & 10 & 10 & 10 & 10 & 10 & 70 & 0 & 197
\end{tblr}
\end{longtblr}

Скориставшись результатами ранжирування, проведемо попарне порівняння всіх параметрів і результати занесемо у таблицю \ref{tbl:pair-comparison}

\begin{longtblr}[
	label=tbl:pair-comparison,
	caption={Попарне порівняння параметрів}
]{
	hlines, vlines,
	columns={c,m,-1},
	cell{1}{1,Y-Z}={r=2}{},
	cell{1}{2}={c=7}{},
	rowhead=2
}
Параметри & Експерти &  &  &  &  &  &  & Кінцева оцінка & Числове значення \\
 & 1 & 2 & 3 & 4 & 5 & 6 & 7 &  &  \\
$X_1$ і $X_2$ & < & < & < & < & < & < & < & < & 0,5 \\
$X_1$ і $X_3$ & < & > & > & < & < & < & > & < & 0,5 \\
$X_1$ і $X_4$ & < & < & < & < & < & < & < & < & 0,5 \\
$X_2$ і $X_3$ & > & > & > & > & > & > & > & > & 1,5 \\
$X_2$ і $X_4$ & < & > & < & < & > & < & < & < & 0,5 \\
$X_3$ і $X_4$ & < & < & < & < & < & < & < & < & 0,5
\end{longtblr}

Числове значення, що визначає ступінь переваги $i$–го параметра над $j$–тим, $a_{ij}$ визначається по формулі:
$$
a_{ij} = 
\begin{cases}
1.5 & при\ X_i>X_j \\
1.0 & при\ X_i=X_j \\
0.5 & при\ X_i<X_j
\end{cases}
$$

З отриманих числових оцінок переваги складемо матрицю $A = ||a_{ij}||$.

Для кожного параметра зробимо розрахунок вагомості Kві за наступними формулами:
\begin{gather}
К_{ві} = \frac{b_i}{\sum_{i=1}^{n}b_i}\\
b_i' = \sum_{i=1}^{N}a_{ij}b_j
\end{gather}

Як видно з таблиці \ref{tbl:weights}, різниця значень коефіцієнтів вагомості не перевищує 2\%, тому більшої кількості ітерацій не потрібно.

\begin{longtblr}[
	label=tbl:weights,
	caption={Розрахунок вагомості параметрів}
]{
	hlines, vlines,
	cell{1}{1}={r=2}{},
	cell{1,Z}{2}={c=4}{},
	cell{1}{Y}={c=2}{},
	cell{1}{W}={c=2}{},
	cell{1}{U}={c=2}{},
	cell{2}{2-Z}={mode=dmath},
	cell{2-Z}{1}={mode=dmath}
}
Параметри $X_i$ & Параметри $X_j$ & & & & Перша ітер. & & Друга ітер. & & Третя ітер. & \\*
 & X_1 & X_2 & X_3 & X_4 & b_i & K_{ві} & b_i^1 & K_{ві}^1 & b_i^2 & K_{ві}^2 \\
X_1 & 1 & 0,5 & 0,5 & 0,5 & 2,5 & 0,16 & 9,25 & 0,16 & 34,125 & 0,16 \\
X_2 & 1,5 & 1 & 1,5 & 0,5 & 4,5 & 0,28 & 16,25 & 0,28 & 59,125 & 0,28 \\
X_3 & 1,5 & 0,5 & 1 & 0,5 & 3,5 & 0,22 & 12,25 & 0,21 & 41,875 & 0,2  \\
X_4 & 1,5 & 1,5 & 1,5 & 1 & 5,5 & 0,34 & 21,25 & 0,35 & 77,875 & 0,36 \\
Всього: &  &  &  &  & 16 & 1 & 59 & 1 & 213 & 1
\end{longtblr}

\section{Аналіз рівня якості варіантів реалізації функцій}

Визначаємо рівень якості кожного варіанту виконання основних функцій окремо.

Абсолютні значення параметрів $Х_2$ (об'єм пам'яті), $X_3$ (час попередньої обробки даних) та $X4$ (потенційний об'єм програмного коду) відповідають технічним вимогам умов функціонування даного ПП.

Абсолютне значення параметра Х1 (швидкість роботи мови програмування) обрано не найгіршим.

Коефіцієнт технічного рівня для кожного варіанта реалізації ПП розраховується так (таблиця \ref{tbl:quality}):

$$
K_K(j) = \sum_{i=1}^{n}К_{ві,j}B_{i,j}
$$
де
\begin{description}
\item[$n$] кількість параметрів; 
\item[$K_{ві}$] коефіцієнт вагомості i–го параметра;
\item[$B_i$] оцінка $i$–го параметра в балах.
\end{description}

\begin{longtblr}[
	label=tbl:quality,
	caption={Розрахунок показників рівня якості варіантів реалізації основних функцій ПП}
]{
	width=\textwidth+0.5cm,
	hlines, vlines,
	columns={c,m,-1,colsep=1pt},
	column{1}={0.8},
	cell{Y}{1}={r=2}{},
	cell{2-Z}{1-Z}={mode=dmath}
}
Основні функції & Варіант реалізації функції & Параметри & Абсолютне значення параметра & Бальна оцінка параметра & Коефіцієнт вагомості параметра & Коефіцієнт рівня якості \\
F_1 & А & Х_1 & 100 & 25 & 0,16 & 4 \\
F_3 & А & Х_2 & 87 & 29 & 0,28 & 8,12 \\
F_4 & Б & X_3 & 27 & 19 & 0,2  & 3,8 \\
    & А & Х_4 & 25 & 23 & 0,36 & 8,28
\end{longtblr}

За даними з таблиці \ref{tbl:quality} за формулою:

$$
K_K = K_{ТУ}[F_{1k}]+K_{ТУ}[F_{2k}]+\dots+K_{ТУ}[F_{zk}]
$$
визначаємо рівень якості кожного з варіантів:
\begin{gather*}
K_{K1} = 4 + 8.12 + 8.28 = 20.4;\\
K_{k2} = 4 + 3.8 + 8.28 = 16.08.
\end{gather*}

Як видно з розрахунків, кращим є 2 варіант, для якого коефіцієнт технічного рівня має найбільше значення.

\section{Економічний аналіз варіантів розробки ПП}

Для визначення вартості розробки ПП спочатку проведемо розрахунок трудомісткості.

Всі варіанти включають в себе два окремих завдання:
\begin{enumerate}
\item Розробка проекту програмного продукту;
\item Розробка програмної оболонки;
\end{enumerate}

Завдання 1 за ступенем новизни відноситься до групи А, завдання 2 -- до групи Б. За складністю алгоритми, які використовуються в завданні 1 належать до групи 1; а в завданні 2 -- до групи 3.

Для реалізації завдання 1 використовується довідкова інформація, а завдання 2 використовує інформацію у вигляді даних.

Проведемо розрахунок норм часу на розробку та програмування для кожного з завдань.

Загальна трудомісткість обчислюється як: 

$$
T_О = T_Р\cdot K_П\cdot K_{СК}\cdot K_М\cdot K_{СТ}\cdot K_{СТ.М}
$$
де
\begin{description}
\item[$T_Р$] трудомісткість розробки ПП;
\item[$K_П$] поправочний коефіцієнт;
\item[$K_П$] коефіцієнт на складність вхідної інформації; 
\item[$K_{СК}$] коефіцієнт рівня мови програмування;
\item[$K_{СТ}$] коефіцієнт використання стандартних модулів і прикладних програм;
\item[$K_{СТ.М}$] коефіцієнт стандартного математичного забезпечення.
\end{description}

Для першого завдання, виходячи із норм часу для завдань розрахункового характеру степеню новизни А та групи складності алгоритму 1, трудомісткість дорівнює: $T_Р = 37$ людино-днів. Поправочний коефіцієнт, який враховує вид нормативно-довідкової інформації для першого завдання: $K_П = 1.8$ Поправочний коефіцієнт, який враховує складність контролю вхідної та вихідної інформації для всіх семи завдань рівний 1: $K_{СК} = 1$. Оскільки при розробці першого завдання використовуються стандартні модулі, врахуємо це за допомогою коефіцієнта $K_{СТ} = 0.9$. Тоді загальна трудомісткість програмування першого завдання дорівнює:
$$
T_2 = 29\cdot 0.9\cdot 0.8 = 20.88 \text{ людино-днів.}
$$

Складаємо трудомісткість відповідних завдань для кожного з обраних варіантів реалізації програми, щоб отримати їх трудомісткість:
\begin{gather*}
Т_I = (59,94  + 20.88 + 4.8 + 20.88)\cdot 8 = 852\ людино-днів. \\
Т_{II} = (59,94  + 20.88 + 6.91 + 20.88)\cdot 8 = 868,88\ людино-днів.
\end{gather*}

Найбільш високу трудомісткість має варіант II.

В розробці беруть участь два програмісти з окладом 17000 грн., один аналітик в області даних з окладом 19000. Визначимо середню зарплату за годину за формулою:
$$
C_Ч = \frac{M}{T_m\cdot t} грн.,
$$
де
\begin{description}
\item[$M$] місячний оклад працівників;
\item[$T_m$] кількість робочих днів тиждень;
\item[$t$] кількість робочих годин в день.
\end{description}

$$
C_Ч = \frac{17000 + 17000 + 19000}{3\cdot 21\cdot 8} грн.,
$$

Тоді, розрахуємо заробітну плату за формулою:
$$
C_{ЗП} = C_Ч\cdot T_i\cdot K_Д
$$
де
\begin{description}
\item[$C_Ч$] величина погодинної оплати праці програміста;
\item[$T_i$] трудомісткість відповідного завдання;
\item[$K_Д$] норматив, який враховує додаткову заробітну плату.
\end{description}

Зарплата розробників за варіантами становить:
\begin{enumerate}[label=\Roman*.]
\item $C_{ЗП} = 105.16\cdot 852\cdot 1.2 = 107515,58~грн.$
\item $C_{ЗП} = 105.16\cdot 868.88\cdot 1.2 = 109645,7~грн.$
\end{enumerate}

Відрахування на єдиний соціальний внесок становить 22\%:
\begin{enumerate}[label=\Roman*.]
\item $C_{ВІД} = C_{ЗП}\cdot 0.22 = 107515,58\cdot 0.22 = 23653,4~грн.$
\item $C_{ВІД} = C_{ЗП}\cdot 0.22 = 109645,7\cdot 0.22 = 24122,06~грн.$
\end{enumerate}

Тепер визначимо витрати на оплату однієї машино-години. ($C_M$)

Так як одна ЕОМ обслуговує одного програміста з окладом 17000 грн., з коефіцієнтом зайнятості 0,2 то для однієї машини отримаємо:
$$
C_Г = 12\cdot M\cdot K_3 = 12\cdot 17000\cdot 0,2 = 40800~грн.
$$

З урахуванням додаткової заробітної плати:
$$
C_{ЗП} = C_Г\cdot (1+K_3) = 40800\cdot (1 + 0.2) = 48960~грн.
$$

Відрахування на соціальний внесок:
$$
C_{ВІД} = C_{ЗП}\cdot 0.22 = 48960\cdot 0,22 = 10771,2~грн.
$$

Амортизаційні відрахування розраховуємо при амортизації 25\% та вартості ЕОМ – 10000 грн.
$$
C_А = K_{ТМ}\cdot K_А\cdot Ц_{ПР} = 1.4\cdot 0.12\cdot 10000 = 1680~грн.,
$$
де
\begin{description}
\item[$K_{ТМ}$] коефіцієнт, який враховує витрати на транспортування та монтаж приладу у користувача;
\item[$K_А$] річна норма амортизації;
\item[$Ц_{ПР}$] договірна ціна приладу.
\end{description}

Витрати на ремонт та профілактику розраховуємо як:
$$
C_Р = K_{ТМ}\cdot Ц_{ПР}\cdot K_Р = 1.4\cdot 10000\cdot 0.08 = 1120~грн.,
$$
де $K_Р$ -- відсоток витрат на поточні ремонти.

Ефективний годинний фонд часу ПК за рік розраховуємо за формулою:
$$
T_{ЕФ} = (Д_К - Д_В - Д_С - Д_Р)\cdot t_3\cdot K_В = (365 - 104 - 12 - 16)\cdot 8\cdot 0.35 = 627,2~години,
$$
де
\begin{description}
\item[$Д_К$] календарна кількість днів у році;
\item[$Д_В, Д_С$] відповідно кількість вихідних та святкових днів;
\item[$Д_Р$] кількість днів планових ремонтів устаткування;
\item[$t$] кількість робочих годин в день;
\item[$K_В$] коефіцієнт використання приладу у часі протягом зміни.
\end{description}

Витрати на оплату електроенергії розраховуємо за формулою:
$$
C_{ЕЛ} = T_{ЕФ}\cdot N_C\cdot K_3\cdot Ц_{ЕН} = 627,2\cdot 0,2\cdot 0,3\cdot 3,52 = 145,7~грн.,
$$
де
\begin{description}
\item[$N_C$] середньо-споживча потужність приладу;
\item[$K_3$] коефіцієнтом зайнятості приладу;
\item[$Ц_{ЕН}$] тариф за 1 КВт-годин електроенергії.
\end{description}

Накладні витрати розраховуємо за формулою:
$$
C_Н = Ц_{ПР}\cdot 0.67 = 10000\cdot 0,67 = 6700~грн.
$$

Тоді, річні експлуатаційні витрати будуть:
\begin{gather*}
C_{ЕКС} = C_{ЗП} + C_{ВІД} + C_{А} + C_{Р} + C_{ЕЛ} + C_{Н} \\
C_{ЕКС} = 48960 + 10771,2 + 1680 + 1120 + 145,7 + 6700 = 69376,90~грн.
\end{gather*}

Собівартість однієї машино-години ЕОМ дорівнюватиме:
$$
C_{М-Г} = C_{ЕКС}/T_{ЕФ} = 69376,90 / 627,2 = 110,60~грн/год.
$$

Оскільки в даному випадку всі роботи, які пов‘язані з розробкою програмного продукту ведуться на ЕОМ, витрати на оплату машинного часу, в залежності від обраного варіанта реалізації, складає:
$$
C_M = C_{М-Г}\cdot T
$$

\begin{enumerate}[label=\Roman*.]
\item $C_M = 110,6\cdot 852 = 94231,2~грн.$
\item $C_M = 110,6\cdot 868,88 = 96098,12~грн.$
\end{enumerate}

Накладні витрати складають 67\% від заробітної плати:
$$
C_Н = C_{ЗП}\cdot 0,67
$$

\begin{enumerate}[label=\Roman*.]
\item $C_Н = 107515,58\cdot 0,67 = 72035,45~грн.$
\item $C_Н = 109645,70\cdot 0,67 = 73462,6~грн.$
\end{enumerate}

Отже, вартість розробки ПП за варіантами становить:
$$
C_{ПП} = C_{ЗП} + C_{ВІД} + C_М + C_Н
$$

\begin{enumerate}[label=\Roman*.]
\item $C_{ПП} = 107515,58 + 23653,4 + 94231,2 + 72035,45 = 297435,63~грн.$
\item $C_{ПП} = 109645,7  + 24122,06 + 96098,12 + 73462,6 = 303328,49~грн.$
\end{enumerate}

\section{Вибір кращого варіанту ПП техніко-економічного рівня}

Розрахуємо коефіцієнт техніко-економічного рівня за формулою:
\begin{gather*}
K_{ТЕРj} = K_{Kj} / C_{фj}, \\
K_{ТЕР1} = 20,4 / 297435,63 = 6,851\cdot 10^{-5}, \\
K_{ТЕР2} = 16,08 / 303328,49 = 5,3011\cdot 10^{-5}.
\end{gather*}

Як бачимо, найбільш ефективним є перший варіант реалізації програми з коефіцієнтом техніко-економічного рівня $K_{ТЕР1} = 6,851\cdot 10^{-5}$.

Після виконання функціонально-вартісного аналізу програмного комплексу що розроблюється, можна зробити висновок, що з альтернатив, що залишились після першого відбору двох варіантів виконання програмного комплексу оптимальним є перший варіант реалізації програмного продукту. У нього виявився найкращий показник техніко-економічного рівня якості $K_{ТЕР} = 6,851\cdot 10^{-5}$.

Цей варіант реалізації програмного продукту має такі параметри:
\begin{itemize}
\item Вибір програмного продукту – Eviews;
\item Реалізація важливої постановки з допомогою вбудованих функцій;
\item Використання стандартного інтерфейсу для побудови значень. 
\end{itemize}

Даний варіант виконання програмного комплексу дає користувачу зручний інтерфейс, швидку реалізацію програми та доступний функціонал для роботи.

\section{Висновки до четвертого розділу}

В даній частині було проведено повний функціонально-вартісний аналіз програмного продукту. Також було знайдено оцінку основних функцій програмного продукту.

В результаті виконання функціонально-вартісного аналізу програмного комплексу що розроблюється, було визначено та проведено оцінку основних функцій програмного продукту, а також знайдено параметри, які його характеризують.

На основі аналізу вибрано варіант реалізації програмного продукту.

\end{document}
